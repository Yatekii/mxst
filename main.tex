\documentclass[10pt,landscape]{article}
\usepackage{multicol}
\usepackage{calc}
\usepackage{ifthen}
\usepackage[landscape]{geometry}
\usepackage{hyperref}

\usepackage{tikz}
\usepackage{amsmath}
\usetikzlibrary{arrows}
\usepackage{float}
\usetikzlibrary{shapes}
\usetikzlibrary{intersections}
\usepackage{circuitikz}

% This sets page margins to .5 inch if using letter paper, and to 1cm
% if using A4 paper. (This probably isn't strictly necessary.)
% If using another size paper, use default 1cm margins.
\ifthenelse{\lengthtest { \paperwidth = 11in}}
	{ \geometry{top=.5in,left=.5in,right=.5in,bottom=.5in} }
	{\ifthenelse{ \lengthtest{ \paperwidth = 297mm}}
		{\geometry{top=1cm,left=1cm,right=1cm,bottom=1cm} }
		{\geometry{top=1cm,left=1cm,right=1cm,bottom=1cm} }
	}

% Turn off header and footer
\pagestyle{empty}
 

% Redefine section commands to use less space
\makeatletter
\renewcommand{\section}{\@startsection{section}{1}{0mm}%
                                {-1ex plus -.5ex minus -.2ex}%
                                {0.5ex plus .2ex}%x
                                {\normalfont\large\bfseries}}
\renewcommand{\subsection}{\@startsection{subsection}{2}{0mm}%
                                {-1explus -.5ex minus -.2ex}%
                                {0.5ex plus .2ex}%
                                {\normalfont\normalsize\bfseries}}
\renewcommand{\subsubsection}{\@startsection{subsubsection}{3}{0mm}%
                                {-1ex plus -.5ex minus -.2ex}%
                                {1ex plus .2ex}%
                                {\normalfont\small\bfseries}}
\makeatother

% Define BibTeX command
\def\BibTeX{{\rm B\kern-.05em{\sc i\kern-.025em b}\kern-.08em
    T\kern-.1667em\lower.7ex\hbox{E}\kern-.125emX}}

% Don't print section numbers
\setcounter{secnumdepth}{0}


\setlength{\parindent}{0pt}
\setlength{\parskip}{0pt plus 0.5ex}


% -----------------------------------------------------------------------

\begin{document}

\raggedright
\footnotesize
\begin{multicols}{2}

% multicol parameters
% These lengths are set only within the two main columns
%\setlength{\columnseprule}{0.25pt}
\setlength{\premulticols}{1pt}
\setlength{\postmulticols}{1pt}
\setlength{\multicolsep}{1pt}
\setlength{\columnsep}{2pt}

\section{Data Conversion Chain | 20-02-2017}

\begin{figure}[H]
    \begin{center}
        \tikzset{font={\fontsize{8pt}{12}\selectfont}}
        \begin{circuitikz}[x=0.021\linewidth,y=0.021\linewidth]

        \draw[name path=chain]
           (0,0) node[above]{$V_{in}$} to[short,o-] ++(0.2, 0)
           to[lowpass,->] ++(6,0)
           to[twoport,t=S\&H] ++(6,0)
           to[adc] ++(6,0)
           to[dsp] ++(6,0)
           to[dac] ++(6,0)
           to[lowpass] ++(6,0)
           to[short,-o] ++(0.2,0) node[above]{$V_{out}$};

        \draw[name path=sndr]
        (36.4,-0.5) to[short] ++(0, -0.5) to[short] ++(3, 0) to[short]
        ++(0,-4) node[draw,below]{SNDR} ++(0, -3) node[align=center]{signal to noise \\ and distortion ratio};

        \draw[name path=cd,dashed] (9.2,-6) node[above,rotate=90]{continuous} node[below,rotate=90]{discrete}
        -- (9.2,6) node{};

        \draw[name path=ad,dashed] (15.2,-6) node[above,rotate=90]{analog} node[below,rotate=90]{digital}
        -- (15.2,6) node{};

        \draw[name path=da,dashed] (27.2,-6) node[above,rotate=90]{digital} node[below,rotate=90]{analog}
        -- (27.2,6) node{};

        \draw[name path=dc,dashed] (33.2,-6) node[above,rotate=90]{discrete} node[below,rotate=90]{continuous}
        -- (33.2,6) node{};

        \end{circuitikz}
    \end{center}
    \caption{A typical signal chain}
    \label{fig:chain}
\end{figure}

\rule{0.3\linewidth}{0.25pt}
\scriptsize

Copyright \copyright\ 2017 Noah Huesser

\end{multicols}
\end{document}
